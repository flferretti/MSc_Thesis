\chapter*{Introduction}

This document is intended to be both an example of the Polimi \LaTeX{} template for Master Theses,
as well as a short introduction to its use. It is not intended to be a general introduction to \LaTeX{} itself,
and the reader is assumed to be familiar with the basics of creating and compiling \LaTeX{} documents (see \cite{oetiker1995not, kottwitz2015latex}). 
\\
The cover page of the thesis must contain all the relevant information:
title of the thesis, name of the Study Programme and School, name of the author,
student ID number, name of the supervisor, name(s) of the co-supervisor(s) (if any), academic year.
The above information are provided by filling all the entries in the command \verb|\puttitle{}|
in the title page section of this template.
\\
Be sure to select a title that is meaningful.
It should contain important keywords to be identified by indexer.
Keep the title as concise as possible and comprehensible even to people who are not experts in your field.
The title has to be chosen at the end of your work so that it accurately captures the main subject of the manuscript. 
\\
Since a thesis might be a substantial document, it is convenient to break it into chapters.
You can create a new chapter as done in this template by simply using the following command
\begin{verbatim}
\chapter{Title of the chapter}
\end{verbatim}
followed by the body text.
\\
Especially for long manuscripts, it is recommended to give each chapter its own file.
In this case, you write your chapter in a separated \verb|chapter_n.tex| file
and then include it in the main file with the following command
\begin{verbatim}
\input{chapter_n.tex}
\end{verbatim}
It is recommended to give a label to each chapter by using the command
\begin{verbatim}
\label{ch:chapter_name}%
\end{verbatim}
where the argument is just a text string that you'll use to reference that part
as follows: \textit{Chapter~\ref{ch:chapter_one} contains \sc{an introduction to}  \dots}.\\
If necessary, an unnumbered chapter can be created by
\begin{verbatim}
\chapter*{Title of the unnumbered chapter}
\end{verbatim}

\chapter{Chapter one}
\label{ch:chapter_one}%
% The \label{...}% enables to remove the small indentation that is generated, always leave the % symbol.

In this chapter additional useful information are reported.

\section{Sections and subsections}
\label{sec:section_name}
Chapters are typically subdivided into sections and subsections, and, optionally,
subsubsections, paragraphs and subparagraphs.
All can have a title, but only sections and subsections are numbered.
A new section is created by the command
\begin{verbatim}
\section{Title of the section}
\end{verbatim}
The numbering can be turned off by using \verb|\section*{}|.
\\
A new subsection is created by the command
\begin{verbatim}
\subsection{Title of the subsection}
\end{verbatim}
and, similarly, the numbering can be turned off by adding an asterisk as follows 
\begin{verbatim}
\subsection*{}
\end{verbatim}

\section{Equations}
\label{sec:eqs}
This section gives some examples of writing mathematical equations in your thesis.

Maxwell's equations read:
\begin{subequations}
    \label{eq:maxwell}
    \begin{align}[left=\empheqlbrace]
    \nabla\cdot \bm{D} & = \rho, \label{eq:maxwell1} \\
    \nabla \times \bm{E} +  \frac{\partial \bm{B}}{\partial t} & = \bm{0}, \label{eq:maxwell2} \\
    \nabla\cdot \bm{B} & = 0, \label{eq:maxwell3} \\
    \nabla \times \bm{H} - \frac{\partial \bm{D}}{\partial t} &= \bm{J}. \label{eq:maxwell4}
    \end{align}
\end{subequations}

Equation~\eqref{eq:maxwell} is automatically labeled by \texttt{cleveref},
as well as Equation~\eqref{eq:maxwell1} and Equation~\eqref{eq:maxwell3}.
Thanks to the \verb|cleveref| package, there is no need to use \verb|\eqref|.
Remember that Equations have to be numbered only if they are referenced in the text.

Equations~\eqref{eq:maxwell_multilabels1}, \eqref{eq:maxwell_multilabels2}, \eqref{eq:maxwell_multilabels3}, and \eqref{eq:maxwell_multilabels4} show again Maxwell's equations without brace:
\begin{align}
    \nabla\cdot \bm{D} & = \rho, \label{eq:maxwell_multilabels1} \\
    \nabla \times \bm{E} +  \frac{\partial \bm{B}}{\partial t} &= \bm{0}, \label{eq:maxwell_multilabels2} \\
    \nabla\cdot \bm{B} & = 0, \label{eq:maxwell_multilabels3} \\
    \nabla \times \bm{H} - \frac{\partial \bm{D}}{\partial t} &= \bm{J} \label{eq:maxwell_multilabels4}.
\end{align}

Equation~\eqref{eq:maxwell_singlelabel} is the same as before,
but with just one label:
\begin{equation}
    \label{eq:maxwell_singlelabel}
    \left\{
    \begin{aligned}
    \nabla\cdot \bm{D} & = \rho, \\
    \nabla \times \bm{E} +  \frac{\partial \bm{B}}{\partial t} &= \bm{0},\\
    \nabla\cdot \bm{B} & = 0, \\
    \nabla \times \bm{H} - \frac{\partial \bm{D}}{\partial t} &= \bm{J}.
    \end{aligned}
    \right.
\end{equation}

\section{Figures, Tables and Algorithms}
Figures, Tables and Algorithms have to contain a Caption that describe their content, and have to be properly reffered in the text.

\subsection{Figures}
\label{subsec:figures}

For including pictures in your text you can use \texttt{TikZ} for high-quality hand-made figures,
or just include them as usual with the command
\begin{verbatim}
\includegraphics[options]{filename.xxx}
\end{verbatim}
Here xxx is the correct format, e.g. \verb|.png|, \verb|.jpg|, \verb|.eps|, \dots.

\begin{figure}[H]
    \centering
    \includegraphics[width=0.3\textwidth]{logo_polimi_scritta.eps}
    \caption{Caption of the Figure to appear in the List of Figures.}
    \label{fig:quadtree}
\end{figure}

Thanks to the \texttt{\textbackslash subfloat} command, a single figure, such as Figure~\ref{fig:quadtree},
can contain multiple sub-figures with their own caption and label, e.g. \color{black} Figure~\ref{fig:polimi_logo1} and Figure~\ref{fig:polimi_logo2}. 

\begin{figure}[H]
    \centering
    \subfloat[One PoliMi logo.\label{fig:polimi_logo1}]{
        \includegraphics[scale=0.5]{Images/logo_polimi_scritta.eps}
    }
    \quad
    \subfloat[Another one PoliMi logo.\label{fig:polimi_logo2}]{
        \includegraphics[scale=0.5]{Images/logo_polimi_scritta2.eps}
    }
    \caption[Shorter caption]{This is a very long caption you don't want to appear in the List of Figures.}
    \label{fig:quadtree2}
\end{figure}


\subsection{Tables}
\label{subsec:tables}

Within the environments \texttt{table} and  \texttt{tabular} you can create very fancy tables as the one shown in Table~\ref{table:example}.
\begin{table}[H]
    \caption*{\textbf{Title of Table (optional)}}
    \centering 
    \begin{tabular}{|p{3em} c c c |}
    \hline
    \rowcolor{bluepoli!40} % comment this line to remove the color
     & \textbf{column 1} & \textbf{column 2} & \textbf{column 3} \T\B \\
    \hline \hline
    \textbf{row 1} & 1 & 2 & 3 \T\B \\
    \textbf{row 2} & $\alpha$ & $\beta$ & $\gamma$ \T\B\\
    \textbf{row 3} & alpha & beta & gamma \B\\
    \hline
    \end{tabular}
    \\[10pt]
    \caption{Caption of the Table to appear in the List of Tables.}
    \label{table:example}
\end{table}

You can also consider to highlight selected columns or rows in order to make tables more readable.
Moreover, with the use of \texttt{table*} and the option \texttt{bp} it is possible to align them at the bottom of the page. One example is presented in Table~\ref{table:exampleC}. 

\begin{table}[H]
\centering 
    \begin{tabular}{|p{3em} | c | c | c | c | c | c|}
    \hline
%    \rowcolor{bluepoli!40}
     & \textbf{column1} & \textbf{column2} & \textbf{column3} & \textbf{column4} & \textbf{column5} & \textbf{column6} \T\B \\
    \hline \hline
    \textbf{row1} & 1 & 2 & 3 & 4 & 5 & 6 \T\B\\
    \textbf{row2} & a & b & c & d & e & f \T\B\\
    \textbf{row3} & $\alpha$ & $\beta$ & $\gamma$ & $\delta$ & $\phi$ & $\omega$ \T\B\\
    \textbf{row4} & alpha & beta & gamma & delta & phi & omega \B\\
    \hline
    \end{tabular}
    \\[10pt]
    \caption{Highlighting the columns}
    \label{table:exampleC}
\end{table}

\begin{table}[H]
\centering 
    \begin{tabular}{|p{3em} c c c c c c|}
    \hline
%    \rowcolor{bluepoli!40}
     & \textbf{column1} & \textbf{column2} & \textbf{column3} & \textbf{column4} & \textbf{column5} & \textbf{column6} \T\B \\
    \hline \hline
    \textbf{row1} & 1 & 2 & 3 & 4 & 5 & 6 \T\B\\
    \hline
    \textbf{row2} & a & b & c & d & e & f \T\B\\
    \hline
    \textbf{row3} & $\alpha$ & $\beta$ & $\gamma$ & $\delta$ & $\phi$ & $\omega$ \T\B\\
    \hline
    \textbf{row4} & alpha & beta & gamma & delta & phi & omega \B\\
    \hline
    \end{tabular}
    \\[10pt]
    \caption{Highlighting the rows}
    \label{table:exampleR}
\end{table}

\subsection{Algorithms}
\label{subsec:algorithms}

Pseudo-algorithms can be written in \LaTeX{} with the \texttt{algorithm} and \texttt{algorithmic} packages.
An example is shown in Algorithm~\ref{alg:var}.
\begin{algorithm}[H]
    \label{alg:example}
    \caption{Name of the Algorithm}
    \label{alg:var}
    \label{protocol1}
    \begin{algorithmic}[1]
    \STATE Initial instructions
    \FOR{$for-condition$}
    \STATE{Some instructions}
    \IF{$if-condition$}
    \STATE{Some other instructions}
    \ENDIF
    \ENDFOR
    \WHILE{$while-condition$}
    \STATE{Some further instructions}
    \ENDWHILE
    \STATE Final instructions
    \end{algorithmic}
\end{algorithm} 

\vspace{5mm}

\section{Theorems, propositions and lists}

\subsection{Theorems}
Theorems have to be formatted as:
\begin{theorem}
\label{a_theorem}
Write here your theorem. 
\end{theorem}
\textit{Proof.} If useful you can report here the proof.

\subsection{Propositions}
Propositions have to be formatted as:
\begin{proposition}
Write here your proposition.
\end{proposition}

\subsection{Lists}
How to  insert itemized lists:
\begin{itemize}
    \item first item;
    \item second item.
\end{itemize}
How to insert numbered lists:
\begin{enumerate}
    \item first item;
    \item second item.
\end{enumerate}

\section{Use of copyrighted material}

Each student is responsible for obtaining copyright permissions, if necessary, to include published material in the thesis.
This applies typically to third-party material published by someone else.

\section{Plagiarism}

You have to be sure to respect the rules on Copyright and avoid an involuntary plagiarism.
It is allowed to take other persons' ideas only if the author and his original work are clearly mentioned.
As stated in the Code of Ethics and Conduct, Politecnico di Milano \textit{promotes the integrity of research,
condemns manipulation and the infringement of intellectual property}, and gives opportunity to all those
who carry out research activities to have an adequate training on ethical conduct and integrity while doing research.
To be sure to respect the copyright rules, read the guides on Copyright legislation and citation styles available
at:
\begin{verbatim}
https://www.biblio.polimi.it/en/tools/courses-and-tutorials
\end{verbatim}
You can also attend the courses which are periodically organized on "Bibliographic citations and bibliography management".

\section{Bibliography and citations}
Your thesis must contain a suitable Bibliography which lists all the sources consulted on developing the work.
The list of references is placed at the end of the manuscript after the chapter containing the conclusions.
We suggest to use the BibTeX package and save the bibliographic references  in the file \verb|Thesis_bibliography.bib|.
This is indeed a database containing all the information about the references. To cite in your manuscript, use the \verb|\cite{}| command as follows:
\\
\textit{Here is how you cite bibliography entries: \cite{knuth74}, or multiple ones at once: \cite{knuth92,lamport94}}.
\\
The bibliography and list of references are generated automatically by running BibTeX \cite{bibtex}.

\chapter{Conclusions and future developments}
\label{ch:conclusions}%
A final chapter containing the main conclusions of your research/study
and possible future developments of your work have to be inserted in this chapter.