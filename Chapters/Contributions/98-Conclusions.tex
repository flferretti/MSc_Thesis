\chapter{Conclusions and Future Work}
\label{chp:contrib_Conclusions}


Although the modified \ac{ABA} algorithm allows for a more realistic forward dynamics computation when there is the need to take into account the motor dynamics, some hypotheses are still made in the derivation of the algorithm. In particular, future versions of the algorithm might take into account the effect of a non-negligible transmission inertia. Furthermore, eliminating the assumption of equal motion subspace for the motor and the link would allow for the computation of more complex kinematic chains involving, for example, internal kinematic loops.

The results obtained with the \ac{RL} framework show that the proposed approach is effective in solving the tasks considered in this work. However, future versions of \jaxsim might include a more effective and accurate contact model, perhaps exploiting the differentiability of \jax, which has not been extensively studied yet. Moreover, the addition of a more accurate contact model and the development of an integrated visualizer would ease its use as a reinforcement learning playground in combination with the emerging differentiable physics engines.
The framework of \jax could be also exploited to have an open-source version of Adversarial Motion Prior, which exploits the differentiability to easily adapt learned policies to new robots. This would allow for a more effective transfer of learning between different robots, which is a crucial aspect in the development of robotic systems.

Finally, the results obtained with the robot codesign process show that the proposed approach is effective in finding a motor combination that is effective in maximizing the reward. However, the results obtained with the genetic algorithm are limited to a discrete motor parameter set. In the future, supposing to have an unlimited choice for motor design, the codesign loop might take into account continuous search spaces for the motor parameters, which would allow for more effective optimization.